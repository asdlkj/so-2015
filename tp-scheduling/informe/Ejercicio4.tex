\section{Ejercicio 4}

\subsection{Desarrollo}

El ejercicio consiste en programar un scheduler de Round-Robin, para esto utilizaremos las siguientes estructuras:
		vector<int> pid_cores;
		vector<bool> cores_bloqueados;
		vector<int> quantum_restantes;
		int cant_cores;
		int cpu_quantum;
		queue<int> enEspera;

Cuando creamos el scheduler se nos dan la cantidad de cores(que será cant_cores) y el quantum de los cpus(cpu_quantum). Segun el enunciado los procesos estan agrupados en 
una unica cola, de ahi viene enEspera. Como conocemos la cantidad de cores usamos tres estructuras de vectores para, solo teniendo el numero del cpu, poder acceder a la información
de las mismas. cores_bloqueados nos indica si el proceso del nucleo en cuestion está o no bloqueado, pid_cores nos da el pid del proceso que esta corriendo en el cpu y
quantum_restantes nos dice cuanto tiempo le queda por ejecutar hasta terminar.

En el programa load en el cual tenemos que poner un proceso en el scheduler lo que haremos será buscar si hay algún core en pid_cores que tenga cargada la tarea IDLE. 
Si lo encontramos, cargamos el proceso en ese core poniendolo en pid_cores y dandole al quantum_restantes de ese proceso el valor de cpu_quantum ya que es una tarea que está
por comenzar a ejecutar. Caso contrario, simplemente lo encolamos en enEspera para ser ejecurada cuando le toque.




\subsection{Experimentación}
