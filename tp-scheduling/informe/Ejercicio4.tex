\section{Ejercicio 4}

\subsection{Desarrollo}
El ejercicio consiste en programar un scheduler de Round-Robin, para esto utilizaremos las siguientes estructuras: vector de enteros pid$\_$cores, vector de enteros pid$\_$bloqueados, 
vector de enteros quantum$\_$restantes, entero cant$\_$cores, entero cpu$\_$quantum, Una cola de enteros enEspera.

Cuando creamos el scheduler se nos dan la cantidad de cores(que será cant$\_$cores) y el quantum de los cpus(cpu$\_$quantum). Según el enunciado los procesos están agrupados 
en una única cola, de ahí viene enEspera. Como conocemos la cantidad de cores usamos tres estructuras de vectores para, solo teniendo el número del cpu, poder acceder a la 
información de las mismas. pid$\_$bloqueados nos indica si el proceso en cuestión está o no bloqueado(Estos pids no están en enEspera ya que esta corresponde a los
ready), pid$\_$cores nos da el pid del proceso que esta corriendo en el cpu y quantum$\_$restantes nos dice cuanto tiempo le queda por ejecutar hasta terminar su tiempo en el 
procesador.

En el programa load en el cual tenemos que poner un proceso en el scheduler lo que haremos será ponerlo en enEspera. En el caso de unblock lo que hacemos es eliminar el pid de 
pid$\_$bloqueados, sabemos que está ahí porque lo añadimos en tick. Por último tenemos el programa tick que se encarga de realizar los procedimientos pertinentes en cada tick de reloj. Tenemos tres casos de motivos:
 
TICK: en el mismo sabemos que ha pasado un tick de reloj y debemos disminuir los quantum que le quedan al procedimiento del cpu. En el caso de que estos terminen en 0 debemos 
plantearnos si debemos desalojar la tarea para poner otra(Si es la IDLE no hacemos nada). De ser el caso, pondremos la tarea actual en la cola enEspera y el próximo elemento
en esta será el nuevo que se hirá ejecutando en este núcleo. Reseteamos el quantum$\_$restantes para que tenga el valor de cpu$\_$quantum. Notece que si no hay más tareas
excepto por la que se estaba ejecutando se volverá a ejecutar la misma.

BLOCK: Si el proceso se va a bloquear añadimos el pid a pid$\_$bloqueados, si existe algún proceso en ready, lo ponemos a ejecutar en ese core. De solo ser un tick en el cual
está bloqueado, solo hacemos lo último(sin añadir a pid$\_$bloqueados).

EXIT: Colocamos en el core la tarea IDLE y si la cola no está vacía, le asignamos la tarea que se encuentre próxima en ella. Después reseteamos los quantum del proceso. 

